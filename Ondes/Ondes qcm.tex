\documentclass[11pt,a4paper]{article}
\usepackage[left=2cm,right=2cm,top=2cm,bottom=3cm]{geometry}
\usepackage{amsmath,amsfonts,amsthm,amssymb,varioref,times}
\usepackage{gensymb}
\usepackage{tikz}
\usepackage{hyperref}
\hypersetup{
    colorlinks=true,
    linkcolor=blue,
    filecolor=magenta,      
    urlcolor=cyan,
}


%to resume numbering in a list
\usepackage{enumitem}

%pour ecrire en français avec les accents
\usepackage[utf8]{inputenc}
\usepackage[T1]{fontenc}
\usepackage{lmodern} % load a font with all the characters
\usepackage{units}

%Image-related packages
\usepackage{wrapfig}
\usepackage{float, graphicx}
\graphicspath{ {./img/} }
\usepackage{subcaption}
\usepackage[export]{adjustbox}


%pour faire des cadres
\usepackage{framed}
\usepackage{xcolor}
\usepackage{tcolorbox}

%chemistry frmulae
\usepackage{chemfig}
\usepackage{chemformula}
 

% parametres des entete et de pieds de pages
\usepackage{fancyhdr}
\pagestyle{fancy}
\fancyhf{}
\lhead{Exercice : Terminale}
\rhead{}
\chead{Ondes \& Vibration}
\rfoot{Page \thepage}
\lfoot{SZayyani}


% pour ecrire sur +sieurs colonnes
\usepackage{multicol}
\setlength{\columnseprule}{0.25pt}
\setlength{\columnsep}{60pt}

% Fusion de lignes de tableaux.
\usepackage{multirow}

% Position verticale des lettres dans la ligne de tableau.
\usepackage{array}


% MATH -----------------------------------------------------------
\newcommand{\To}{\longrightarrow}
\newcommand{\gpl}{\; g\cdot L^{-1}}
\newcommand{\gpmol}{\; g\cdot mol^{-1}}
\newcommand{\mpl}{\; mol\cdot L^{-1}}
\newcommand{\mps}{\; m\cdot s^{-1}}
\newcommand{\mpss}{\; m\cdot s^{-2}}
\newcommand{\es}[1]{\cdot10^{#1}}
\newcommand{\eng}[1]{\textcolor{purple}{= #1}}
\newenvironment{eg}
 {\begin{shaded} \textbf{Exemples:} } { \end{shaded}}

\newcounter{exo}
\newenvironment{exo}
 {\refstepcounter{exo} \begin{shaded}\noindent $\triangleright \quad$\textbf{Exercice: } } { \end{shaded}}    

\newenvironment{defn}[1]
 {\begin{leftbar}\noindent \textbf{Définition :\textit{ \quad #1} } } { \end{leftbar}} 
\newenvironment{rmrq}
 {\begin{shaded} \textbf{Remarque: Pour aller plus loin ...}\\ \itshape } { \end{shaded}}

\definecolor{shadecolor}{gray}{0.85}

\title{QCM - Ondes}
\date{}
\author{}

\setlength{\parindent}{0mm}
\setlength{\parskip}{2mm}

\begin{document}

\section*{Exercice}
Choisir la/les confirmation(s) correcte(s) pour chaque question. (1 point pour chaque bonne confirmation, -0,5 point pour chaque confirmation fausse). 

\begin{enumerate}
    \item 	\textbf{La vitesse des ondes sonores dans l’air à $20 \degree C$ est de $340 \mps$. Les ondes sonores audibles par l’oreille humaine sont caractérisées par: }
    \begin{enumerate}
        \item 	des fréquences de 20 $Hz$ à 20 $kHz$. 
	    \item des longueurs d’ondes de 17 $mm$ à 17 $m$. 
        \item des périodes de 50 $\mu s$ à 500 $ms$. 
    \end{enumerate}
    \item \textbf{Plus la fréquence d’une onde acoustique de faible amplitude se propageant dans un milieu homogène est élevée: }
    \begin{enumerate}
        \item plus son timbre est élevé
        \item plus sa période est grande
        \item plus sa longueur d’onde est grande
    \end{enumerate}
    \item 	\textbf{L’intensité sonore correspondant au seuil d’audibilité est de $1,0\es{-12}\; W\cdot m^{-2}$. Une trompette est située à une distance $d$ d’un auditeur. Le son émis par cet instrument de musique est perçu par l’auditeur avec une intensité de $1,0\es{-5}\; W\cdot m^{-2}$. }
    \begin{enumerate}
        \item le niveau sonore du son émis est $L=70\; dB$
        \item le son émis par deux trompettes identiques, placées à la distance 2d , est perçu par l’auditeur avec une intensité sonore de $1,0\es{-5}\; W\cdot m^{-2}$
        \item L’auditeur perçoit le son émis par les deux instruments avec un niveau sonore de $73\; dB$
    \end{enumerate}
    \item 	\textbf{Un sonomètre placé devant un haut-parleur mesure un niveau sonore de $75\; dB$. L’intensité sonore au seuil d’audibilité est $1,0\es{-12}\; W\cdot m^{-2}$. }
    \begin{enumerate}
        \item 	l’intensité sonore au niveau du sonomètre est $3,2\es{-5}\; W$
        \item l’intensité sonore au niveau du sonomètre est $3,2\es{-5}\; W\cdot m^{-2}$
        \item le niveau sonore augmente de $13\; dB$ l’intensité sonore est 20 fois supérieure
    \end{enumerate}
    \item \textbf{L’intensité sonore d’une guitare mesurée à une distance de 10 mètre est $I=1,0\es{-5}\; W\cdot m^{-2}$. }
    \begin{enumerate}
        \item Le niveau sonore de la même guitare mesuré à une distance de $15\; m$ est $L=66\; dB$
        \item L’intensité sonore de 3 guitares identiques mesurée à une distance de $15\; m$ est $I=1,3\es{-5}\; W\cdot m^{-2}$. 
        \item Le niveau sonore de 4 guitares identiques à une distance de $20\; m$ est $L=70\; dB$

    \end{enumerate}
\end{enumerate}    
    
\section*{Corrigé}

\begin{multicols}{2}
\begin{enumerate}
    \item (a) et (b) 
    \item aucune
    \item (a) et (c)
    \item (b) et (c)
    \item (a) et (b) et (c)
\end{enumerate}
\end{multicols}















\end{document}