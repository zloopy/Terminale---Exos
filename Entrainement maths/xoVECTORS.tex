\documentclass[12pt,a4paper,fleqn]{article}
\usepackage[left=1.8cm,right=1.8cm,top=2cm,bottom=1.5cm]{geometry}
\usepackage{amsmath,amsfonts,amsthm,amssymb,varioref,times}
\usepackage{gensymb}
\usepackage{lipsum}

%enumerate pacakge 
\usepackage{enumerate}

%pour ecrire en français avec les accents
\usepackage[utf8]{inputenc}
\usepackage[T1]{fontenc}
\usepackage{lmodern} % load a font with all the characters
\usepackage{pifont}

% Divers ...
\usepackage[thmmarks,amsmath]{ntheorem}
\usepackage{fancybox}
\usepackage{frcursive}
\usepackage{pstricks,pstricks-add, pst-plot}
\usepackage[crop=off]{auto-pst-pdf}
%pour les entêtes et pieds de pages
\usepackage{layout}



%Image-related packages
\usepackage{graphicx}
\graphicspath{ {./img/} }
\usepackage{subcaption}
\usepackage[export]{adjustbox}
\usepackage{wrapfig}

%pour les maths
\usepackage{amsmath,amsfonts,amssymb}

%pour faire des cadres
\usepackage{framed}
\usepackage{xcolor}
\usepackage{tikz}
\usepackage{tkz-euclide}
\usetkzobj{all}

% parametres des entete et de pieds de pages
\usepackage{fancyhdr}
\pagestyle{fancy}
\fancyhf{}
\lhead{Exercices : Analyse vectorielle}
\rhead{Physique : Terminale}
\rfoot{Page \thepage}


% pour ecrire sur +sieurs colonnes
\usepackage{multicol}
% \setlength{\columnseprule}{0.25pt}
\setlength{\columnsep}{60pt}

% Fusion de lignes de tableaux.
\usepackage{multirow}
% Position verticale des lettres dans la ligne de tableau.
\usepackage{array}

% MATH -----------------------------------------------------------
\newcommand{\norm}[1]{\left\Vert#1\right\Vert}
\newcommand{\abs}[1]{\left\vert#1\right\vert}
\newcommand{\set}[1]{\left\{#1\right\}}
\newcommand{\Real}{\mathbb R}
\newcommand{\eps}{\varepsilon}
\newcommand{\To}{\longrightarrow}
\newcommand{\BX}{\mathbf{B}(X)}
\newcommand{\A}{\mathcal{A}}
\newcommand{\mps}{m\cdot s^{-1}}
\newcommand{\mpss}{m\cdot s^{-2}}
%repère
\newcommand{\oij} % (\text{O} \, ; \vec{\imath}, \, ,\vec{\jmath}$}


\begin{document}
\title{Exercices : Vecteurs}
\date{}
\maketitle

\section{Decomposition vectorielle}

\subsection*{Déterminer les force ou l'angle qui manquent : }

\begin{multicols}{2}
    \begin{enumerate}[i]
        \item \begin{tikzpicture}[scale=1.3]
    % Draw axes
    \draw [<->] (0,-1) -- (0,2) node (yaxis) [above] {$y$} ;
    \draw [<->] (-1, 0) -- (2,0) node (xaxis) [right] {$x$};
    \draw [->, thick] (0,0) -- (1.5,1.5) ;
    \draw [->, ultra thick] (0,0) -- (1.5,0) ; 
    \draw (1,0) node [below]{$\vec{F_x}$} ;
    \draw [->, ultra thick] (0,0) -- (0,1.5) ; 
    \draw (0,1) node [left]{$\vec{F_y}$} ; 
    \draw (1, 1) node [above, rotate=45]{$\Vec{F}=10 N$} ;
    \draw (0.5,0) arc (0:45:0.5) ;
    \draw (20:1.2) node {$\theta=30\degree$};
    \node[draw,text width=1.5cm] at (3,1) { $\vec{F_x} = $ \\ $\Vec{F_y}=$ } ;
    \end{tikzpicture}

        \item \begin{tikzpicture}[scale=1.3]
    % Draw axes
    \draw [<->] (0,-1) -- (0,2) node (yaxis) [above] {$y$} ;
    \draw [<->] (-2, 0) -- (1,0) node (xaxis) [left] {$x$};
    \draw [->, thick] (0,0) -- (-1.5,1.5) ;
    \draw [->, ultra thick] (0,0) -- (-1.5,0) ; 
    \draw (-1,0) node [below]{$\vec{F_x}$} ;
    \draw [->, ultra thick] (0,0) -- (0,1.5) ; 
    \draw (0,1) node [right]{$\vec{F_y}$} ; 
    \draw (-1, 1) node [above, rotate=-45]{$\Vec{F}=20 N$} ;
    \draw (-0.25,0.25) arc (135:180:0.3) ;
    \draw (160:1) node {$\theta=60\degree$};
    \node[draw,text width=1.5cm] at (2,1) {$\vec{F_x} = \quad $\\ $\vec{F_y}=$};

    
\end{tikzpicture}
        \item \begin{tikzpicture}[scale=1.3]
    % Draw axes
    \draw [<->] (0,-1) -- (0,2) node (yaxis) [above] {$y$} ;
    \draw [<->] (-1, 0) -- (2,0) node (xaxis) [right] {$x$};
    \draw [->, thick] (0,0) -- (1.5,1.5) ;
    \draw [->, ultra thick] (0,0) -- (1.5,0) ; 
    \draw (1,0) node [below]{$\vec{F_x}=5N $} ;
    \draw [->, ultra thick] (0,0) -- (0,1.5) ; 
    \draw (0,1) node [left]{$\vec{F_y}$} ; 
    \draw (1, 1) node [above, rotate=45]{$\Vec{F}$} ;
    \draw (0.5,0) arc (0:45:0.5) ;
    \draw (20:1.2) node {$\theta=45\degree$};
    \node[draw,text width=1.5cm] at (3,1) {$\vec{F} = $\\ $\vec{F_y}=$};
    \end{tikzpicture}

        \item \begin{tikzpicture}[scale=1.3]
    % Draw axes
    \draw [<->] (0,-2) -- (0,1) node (yaxis) [above] {$y$} ;
    \draw [<->] (-2, 0) -- (1,0) node (xaxis) [right] {$x$};
    \draw [->, thick] (0,0) -- (-1.5,-1.5) ;
    \draw [->, ultra thick] (0,0) -- (-1.5,0) ; 
    \draw (-1,0) node [above]{$\vec{F_x}$} ;
    \draw [->, ultra thick] (0,0) -- (0,-1.5) ; 
    \draw (0,-1) node [right]{$\vec{F_y}=10N$} ; 
    \draw (-0.25,-.25) arc (235:270:0.45) ;
    \draw (290:0.5) node {$\theta=66\degree$};
    \node[draw,text width=1.5cm] at (1.5,1) {$\Vec{F_x} = $\\ $\Vec{F}= $};
\end{tikzpicture}    

 \item \begin{tikzpicture}[scale=1.3]
    % Draw axes
    \draw [<->] (0,-1) -- (0,2) node (yaxis) [above] {$y$} ;
    \draw [<->] (-1, 0) -- (2,0) node (xaxis) [right] {$x$};
    \draw [->, thick] (0,0) -- (1,1.5) ;
    \draw [->, ultra thick] (0,0) -- (1,0) ; 
    \draw (1,0) node [below]{$\vec{F_x}=10$} ;
    \draw [->, ultra thick] (0,0) -- (0,1.5) ; 
    \draw (0,1) node [left]{$\vec{F_y}=15N$} ; 
    \draw (0.5, 1) node [above, rotate=55]{$\Vec{F}$} ;
    \draw (0.5,0) arc (0:55:0.5) ;
    \draw (22.5:1) node {$\theta$};
    \node[draw,text width=1.5cm] at (2.5, 1) { $\Vec{F} = $ \\ $\theta=$ } ;
    \end{tikzpicture}

        \item \begin{tikzpicture}[scale=1.3]
    % Draw axes
    \draw [<->] (0,-1) -- (0,2) node (yaxis) [above] {$y$} ;
    \draw [<->] (-2, 0) -- (1,0) node (xaxis) [left] {$x$};
    \draw [->, thick] (0,0) -- (-1,1.5) ;
    \draw [->, ultra thick] (0,0) -- (-1,0) ; 
    \draw (-0.75,0) node [below]{$\vec{F_x}=20N$} ;
    \draw [->, ultra thick] (0,0) -- (0,1.5) ; 
    \draw (0,1) node [right]{$\vec{F_y}$} ; 
    \draw (-0.75, 1.25) node [above, rotate=-55]{$\Vec{F}=100 N$} ;
    \draw (150:0.5) node {$\theta$};
    \node[draw,text width=1.5cm] at (2,1) {$\theta = $ \\ $\vec{F_y}=$};
\end{tikzpicture}    

    \end{enumerate}

\end{multicols}

\newpage

\section{Addition vectorielle.}
\subsection*{Caractériser le vecteur-force résultante.}
\begin{multicols}{2}
\begin{enumerate}[i]

\item \begin{tikzpicture}[scale=1.3]
    % Draw axes
    \draw [<->] (0,-1) -- (0,2) node (yaxis) [above] {$y$} ;
    \draw [<->] (-1, 0) -- (2,0) node (xaxis) [right] {$x$};
    % Draw vectors    
    \draw [->, ultra thick] (0,0) -- (1.5,0) ; 
    \draw (1,0) node [below]{$\vec{F_1}=10N$} ;
    \draw [->, ultra thick] (1.5,0) -- (1.5,2) ; 
    \draw (1.5,1) node [right]{$\vec{F_2}=25N$} ; 

    \end{tikzpicture}

\item \begin{tikzpicture}[scale=1.3]
    % Draw axes
    \draw [<->] (0,-1) -- (0,2) node (yaxis) [above] {$y$} ;
    \draw [<->] (-2, 0) -- (2,0) node (xaxis) [right] {$x$};
    % Draw vectors    
    \draw [->, ultra thick] (0,0) -- (1.5,1) ; 
    \draw (1.5, 1) node [above]{$\Vec{F_1}=10N$} ;
    \draw [->, ultra thick] ((0,0) -- (-1,1.5) ; 
    \draw (-1,1.5) node [above]{$\vec{F_2}=10N$} ; 
    \draw (0.5, 0.2) node [right]{$30\degree$} ;
    \draw (-0.85, 0.2) node [right]{$60\degree$} ;

    \end{tikzpicture}

\item \begin{tikzpicture}[scale=1.3]
    % Draw axes
    \draw [<->] (0,-2) -- (0,2) node (yaxis) [above] {$y$} ;
    \draw [<->] (-2, 0) -- (2,0) node (xaxis) [right] {$x$};
    % Draw vectors    
    \draw [->, ultra thick] (0,0) -- (1.5,1.5) ; 
    \draw (1.5, 1.5) node [above]{$\Vec{F_1}=15N$} ;
    \draw [->, ultra thick] ((0,0) -- (-2,0) ; 
    \draw (-1.5,0.5) node [above]{$\vec{F_2}=20N$} ;
    \draw [->, ultra thick] ((0,0) -- (1.5,-1) ; 
    \draw (1.5,-1) node [below]{$\vec{F_3}=20N$} ;
    \draw (0.5, 0.25) node [right]{$45\degree$} ;
    \draw (0.5, -0.25) node [right]{$60\degree$} ;

    \end{tikzpicture}

\item \begin{tikzpicture}[scale=1.3]
    % Draw axes
    \draw [<->] (0,-2) -- (0,2) node (yaxis) [above] {$y$} ;
    \draw [<->] (-2, 0) -- (2,0) node (xaxis) [right] {$x$};
    % Draw vectors    
    \draw [->, ultra thick] (0,0) -- (1.5,1) ; 
    \draw (2, 1) node [above]{$\Vec{F_1}=100N$} ;
    \draw (0.5, 0.25) node [right]{$33\degree$} ;

    \draw [->, ultra thick] ((0,0) -- (-1.5,-1.5) ; 
    \draw (-1.5, -1.5) node [below]{$\vec{F_2}=50N$} ;
    \draw (-0.5, -0.25) node [left]{$50\degree$} ;
    
    \draw [->, ultra thick] ((0,0) -- (1.5,2) ; 
    \draw (1,2) node [above]{$\vec{F_3}=200N$} ;
    \draw (0.25,0.75) node [above]{$60\degree$} ;

    \end{tikzpicture}

\item \begin{tikzpicture}[scale=1.3]
    % Draw axes
    \draw [<->] (0,-2) -- (0,2) node (yaxis) [above] {$y$} ;
    \draw [<->] (-2, 0) -- (2,0) node (xaxis) [right] {$x$};
    % Draw vectors    
    \draw [->, ultra thick] (0,0) -- (1,1.5) ; 
    \draw (1,1.5) node [above]{$\Vec{F_1}=10N$} ;
    \draw (0.25, 0.55) node [above]{$30\degree$} ;

    \draw [->, ultra thick] ((0,0) -- (-1.5,-0.5) ; 
    \draw (-1.5,-0.35) node [below]{$\vec{F_2}=10N$} ;
    \draw (-1, -0.25) node [left]{$10\degree$} ;
    
    \draw [->, ultra thick] (0,0) -- (1,-1.5) ; 
    \draw (1,-1.5) node [below]{$\vec{F_3}=10N$} ;
    \draw (0.5,-0.35) node [right]{$30\degree$} ;

    \draw [->, ultra thick] (0,0) -- (2,0) ; 
    \draw (1,0.25) node [right]{$\vec{F_4}=10N$} ;
    \draw (0.5,-0.25) node [right]{} ;
    
    \end{tikzpicture}

\item \begin{tikzpicture}[scale=1.3]
    % Draw axes
    \draw [<->] (0,-3) -- (0,0.25) node (yaxis) [above] {$y$} ;
    \draw [<->] (-0.25, 0) -- (3.5,0) node (xaxis) [above] {$x$};
    % Draw vectors    
    \draw [->, ultra thick] (0,0) -- (0.75,-1.5) ; 
    \draw (0.5,-0.75) node [left]{$\Vec{F_1}=20N$} ;
    \draw (0.25, -0.25) node [right]{$60\degree$} ;

    \draw [->, ultra thick] (0.75,-1.5) -- (2,-2) ; 
    \draw (2,-2.25) node [right]{$\vec{F_2}=20N$} ;
    
    \draw [->, ultra thick] (2,-2) -- (3,-1) ; 
    \draw (2.5,-0.5) node [right]{$\vec{F_3}=15N$} ;

    \draw [dotted, thick] (0.75,-2.5) -- (0.75,-1) ;
    \draw (0.8, -2) node [right]{$60\degree$};
    \draw [dotted, thick] (2, -2) -- (3.5, -2) ;
    \draw (2.35, -1.75) node [right]{$30\degree$}
    
    \end{tikzpicture}

\end{enumerate}    
\end{multicols}

\newpage

\section{Addition vectorielle.}
\subsection*{Caractériser le vecteur-force $F$ qui donnerait une somme vectorielle nulle.}
\begin{multicols}{2}
\begin{enumerate}[i]

\item \begin{tikzpicture}[scale=1.3]
    % Draw axes
    \draw [<->] (0,-2) -- (0,2) node (yaxis) [above] {$y$} ;
    \draw [<->] (-2, 0) -- (2,0) node (xaxis) [right] {$x$};
    % Draw vectors    
    \draw [->, ultra thick] (0,0) -- (1.75,1) ; 
    \draw (1.5,1.25) node [above]{$\Vec{F_1}=10N$} ;
    \draw (0.5, 0.25) node [right]{$20\degree$} ;

    \draw [->, ultra thick] (0,0) -- (-1,-1.5) ; 
    \draw (-1,-1.5) node [below]{$\vec{F_2}=20N$} ;
    \draw (-.25, -0.5) node [below]{$20\degree$} ;
    
    \draw [->, ultra thick] (0,0) -- (-1,1) ; 
    \draw (-0.35,1.25) node [left]{$\vec{F}=$} ;
    %\draw (0.5,-0.35) node [right]{$30\degree$} ;

    \end{tikzpicture}

\item \begin{tikzpicture}[scale=1.3]
    % Draw axes
    \draw [<->] (0,-2) -- (0,2) node (yaxis) [above] {$y$} ;
    \draw [<->] (-2, 0) -- (2,0) node (xaxis) [right] {$x$};
    % Draw vectors    
    \draw [->, ultra thick] (0,0) -- (1.75,1) ; 
    \draw (1.5,1.25) node [above]{$\Vec{F_1}=25N$} ;
    \draw (0.55, 0.25) node [right]{$25\degree$} ;

    \draw [->, ultra thick] (0,0) -- (0,-1.5) ; 
    \draw (0,-1.5) node [right]{$\vec{F}$} ;
    % \draw (-.25, -0.5) node [below]{$20\degree$} ;
    
    \draw [->, ultra thick] (0,0) -- (-1.5,1.5) ; 
    \draw (-1.5, 1.5) node [above]{$\vec{F_2}=30$} ;
    \draw (-0.35, 0.25) node [left]{$45\degree$} ;

    \end{tikzpicture}

\item \begin{tikzpicture}[scale=1.3]
    % Draw axes
    \draw [<->] (0,-2) -- (0,2) node (yaxis) [above] {$y$} ;
    \draw [<->] (-2, 0) -- (2,0) node (xaxis) [right] {$x$};
    % Draw vectors    
    \draw [->, ultra thick] (0,0) -- (1.5,0) ; 
    \draw (1,0.35) node [right]{$\Vec{F_2}=100N$} ;
    %\draw (0.55, 0.25) node [right]{$25\degree$} ;

    \draw [->, ultra thick] (0,0) -- (0,-1.5) ; 
    \draw (0,-1.5) node [right]{$\vec{F}$} ;
    %\draw (-.25, -0.5) node [below]{$20\degree$} ;
    
    \draw [->, ultra thick] (0,0) -- (-1.5,1) ; 
    \draw (-1.5, 1) node [above]{$\vec{F_1}=30N$} ;
    \draw (-0.5, 0.25) node [left]{$30\degree$} ;

    \end{tikzpicture}

\item \begin{tikzpicture}[scale=1.3]
    % Draw axes
    \draw [<->] (0,-2) -- (0,2) node (yaxis) [above] {$y$} ;
    \draw [<->] (-2, 0) -- (2,0) node (xaxis) [right] {$x$};
    % Draw vectors    
    \draw [->, ultra thick] (0,0) -- (2,0.75) ; 
    \draw (1.75,1) node [right]{$\Vec{F_1}=10N$} ;
    \draw (0.8, 0.2) node [right]{$15\degree$} ;

    \draw [->, ultra thick] (0,0) -- (-0.5,-1.5) ; 
    \draw (-0.5,-1.5) node [below]{$\vec{F}$} ;
    % \draw (-.25, -0.5) node [below]{$20\degree$} ;
    
    \draw [->, ultra thick] (0,0) -- (-1,1.5) ; 
    \draw (-1, 1.5) node [above]{$\vec{F_2}=10$} ;
    \draw (-0.25, 0.25) node [left]{$60\degree$} ;
    
    \draw [->, ultra thick] (0,0) -- (0.75,2) ; 
    \draw (0.75,2) node [right]{$\Vec{F_3}=10N$} ;
    \draw (0.25,1) node [above]{$15\degree$} ;

    \end{tikzpicture}

\end{enumerate}
\end{multicols}

\newpage

\section{Avec un repère "incliné"}
\subsection*{Déterminer la somme des vecteurs et, si nécessaire, caractériser la force nécessaire pour équilibrer l'ensemble.}

\begin{enumerate}[i]
    \item 
    \begin{center}
       \begin{tikzpicture}[scale=1.3]
    % draw inclined plane
    \draw  (0,0) -- (7,0) ;
    \draw  (2,0) -- (6,2) ; 
    \draw (2.5, 0.15) node [right]{$\theta = 30\degree$} ;
    \draw (2,0) node [below]{$A$} ;
    
    %forces
    \draw [->, ultra thick] (4, 1.25) -- (4, -0.5) ; 
    \draw (4,-0.5) node [below]{$\Vec{F_g}=50N$} ;
    
    \draw [->, ultra thick] (4, 1.25) -- (5, 1.75) ; 
    \draw (4.75, 1.75) node [above, rotate=30] {$\Vec{F_f}=10N$} ; 
   
    \draw [->, ultra thick] (4, 1.25) -- (3.35, 2.5) ; 
    \draw (3.35, 2.5) node [above]{$\Vec{F_N}=40$} ; 
    
    \coordinate (O) at (4, 1.25) {}; 
    %base
    \coordinate (B) at (3.35, 2.5) {}; 
    \coordinate (C) at (5, 1.75) {}; 
    
    \tkzMarkRightAngle[draw=black, size=.2](B,O,C) ;
    
    
    \end{tikzpicture}
    \end{center}

    \item 
    \begin{center}
       \begin{tikzpicture}[scale=1.3]
    % draw inclined plane
    \draw  (0,0) -- (7,0) ;
    \draw  (2,0) -- (6,2) ; 
    \draw (2.5, 0.15) node [right]{$\theta = 30\degree$} ;
    \draw (2,0) node [below]{$A$} ;
    
    %forces
    \draw [->, ultra thick] (4, 1) -- (4, -0.5) ; 
    \draw (4,-0.5) node [below]{$\Vec{F_g}=30N$} ;
    
    \draw [->, ultra thick] (4, 1) -- (5.5, 3) ; 
    \draw (5.5, 3) node [above] {$\Vec{F_1}=50N$} ; 
    \draw (4.5, 1.5) node [right, rotate=27] {$\theta=30$} ;
   
    \draw [->, ultra thick] (4, 1) -- (3.25, 2.5) ; 
    \draw (3.25, 2.5) node [above]{$\Vec{F_N}=20N$} ; 
    
    \draw [->, ultra thick] (4, 1) -- (3, 0.5) ; 
    \draw (3, 0.6) node [above, rotate=27]{$\Vec{F_2}=10N$} ;
    
    
    
    
    \coordinate (O) at (4, 1) {}; 
    \coordinate (B) at (3.25, 2.5) {}; 
    \coordinate (A) at (2,0) {}; 
    
    \tkzMarkRightAngle[draw=black, size=.2](A,O,B) ;
    
    
    \end{tikzpicture}
    \end{center}
   
 
    
    
\end{enumerate}




\end{document}