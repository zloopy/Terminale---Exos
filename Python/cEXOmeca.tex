\documentclass[11pt,a4paper]{article}
\usepackage[left=2cm,right=2cm,top=2cm,bottom=3cm]{geometry}
\usepackage{amsmath,amsfonts,amsthm,amssymb,varioref,times}
\usepackage{gensymb}
\usepackage{tikz}
\usepackage{hyperref}
\hypersetup{
    colorlinks=true,
    linkcolor=blue,
    filecolor=magenta,      
    urlcolor=cyan,
}
\usepackage{xhfill}
%pour ecrire en français avec les accents
\usepackage[utf8]{inputenc}
\usepackage[T1]{fontenc}
\usepackage{lmodern} % load a font with all the characters
\usepackage{units}

% for writing code
\usepackage{listings}
\usepackage{minted}
\definecolor{codegreen}{rgb}{0,0.6,0}
\definecolor{codegray}{rgb}{0.5,0.5,0.5}
\definecolor{codepurple}{rgb}{0.58,0,0.82}
\definecolor{backcolour}{rgb}{0.95,0.95,0.92}
\lstdefinestyle{mystyle}{
    backgroundcolor=\color{backcolour},   
    commentstyle=\color{codegreen},
    keywordstyle=\color{magenta},
    numberstyle=\tiny\color{codegray},
    stringstyle=\color{codepurple},
    basicstyle=\ttfamily\footnotesize,
    breakatwhitespace=false,         
    breaklines=true,                 
    captionpos=b,                    
    keepspaces=true,                 
    numbers=left,                    
    numbersep=5pt,                  
    showspaces=false,                
    showstringspaces=false,
    showtabs=false,                  
    tabsize=2
}

\lstset{style=mystyle}

%to resume numbering in a list
\usepackage{enumitem}

%Image-related packages
\usepackage{wrapfig}
\usepackage{float, graphicx}
\graphicspath{ {./img/} }
\usepackage{subcaption}
\usepackage[export]{adjustbox}


%pour faire des cadres
\usepackage{framed}
\usepackage{xcolor}
\usepackage{tcolorbox}


% parametres des entete et de pieds de pages
\usepackage{fancyhdr}
\pagestyle{fancy}
\fancyhf{}
\lhead{SciPhy : 2nde}
\rhead{Ch. 6bis : Gravitation}
\chead{2019}
\rfoot{Page \thepage}
\lfoot{SZayyani}


% pour ecrire sur +sieurs colonnes
\usepackage{multicol}
\setlength{\columnseprule}{0.25pt}
\setlength{\columnsep}{60pt}

% Fusion de lignes de tableaux.
\usepackage{multirow}

% Position verticale des lettres dans la ligne de tableau.
\usepackage{array}


% MATH -----------------------------------------------------------
\newcommand{\To}{\longrightarrow}
\newcommand{\gpl}{\; g\cdot L^{-1}}
\newcommand{\gpmol}{\; g\cdot mol^{-1}}
\newcommand{\mpl}{\; mol\cdot L^{-1}}
\newcommand{\mps}{\; m\cdot s^{-1}}
\newcommand{\mpss}{\; m\cdot s^{-2}}
\newcommand{\es}[1]{\cdot10^{#1}}
\newcommand{\eng}[1]{\textcolor{purple}{= #1}}
\newenvironment{eg}
 {\begin{shaded} \textbf{Exemples:} } { \end{shaded}}

\newcounter{exo}
\newenvironment{exo}
 {\refstepcounter{exo} \begin{shaded}\noindent $\triangleright \quad$\textbf{Exercice: } } { \end{shaded}}    

\newenvironment{defn}[1]
 {\begin{leftbar}\noindent \textbf{Définition :\textit{ \quad #1} } } { \end{leftbar}} 
\newenvironment{python}
 {\begin{shaded} \textbf{A faire en PYTHON}\\ \itshape } { \end{shaded}}

\definecolor{shadecolor}{gray}{0.9}
% %%%%%%%%%%%%%%%%%%%%%for section headers with line goign to the end 

\usepackage[explicit]{titlesec}
    % Raised Rule Command:
    % Arg 1 (Optional) - How high to raise the rule
    % Arg 2 - Thickness of the rule
    \newcommand{\raisedrulefill}[2][0ex]{\leaders\hbox{\rule[#1]{1pt}{#2}}\hfill}
    \titleformat{\section}{\Large\bfseries}{\thesection. }{0em}{#1\,\raisedrulefill[0.4ex]{2pt}}
    %%%%%%%%%%%%%%%%%
\title{Corrigé PYTHON - mécanique newtonienne }
\date{}
\author{}

\setlength{\parindent}{0mm}
\setlength{\parskip}{2mm}

\begin{document}
\maketitle

\section{Choc entre deux boules}
Voici donc le code pour le premier exo. 

\begin{lstlisting}[language=Python]
import numpy as np

m1 = float(input("Choisir la masse de la première boule (en kg) : "))
m2 = float(input("Choisir la masse de la deuxième boule (en kg): "))
u1 = float(input("Choisir la vitesse initiale de la première boule (en m/s) : "))
v1 = float(input("Choisir la vitesse initiale de la deuxième boule (en m/s) : "))

gauche = np.array([[m1,m2],[1,-1]])
droite = np.array([m1*u1 + m2*v1, v1-u1])

soln = np.linalg.solve(gauche, droite)

u2 = soln[0]
v2 = soln[1]

print('u2 = ', u2)
print('v2 = ', v2)

\end{lstlisting}

\section{Collision entre deux voitures (en deux dimensions)}

Le principe ici c'est tout simplement une addition vectorielle des deux vecteur-vitesse des deux voitures pour donner le vecteur-vitesse de l'ensemble des deux voiture collées après la collision.

\begin{lstlisting}[language=python]

import math as m
import numpy as np


m1 = float(input("Choisir la masse du premier véhicule (en kg) : "))
m2 = float(input("Choisir la masse du deuxième véhicule (en kg): "))
u1 = float(input("Choisir la vitesse du premier véhicule (en m/s) : "))
theta1 = float(input("Choisir l'angle du vecteur-vitesse du premier véhicule p/r à l'horizontal' (en degré) : "))
u2 = float(input("Choisir la vitesse du deuxième véhicule (en m/s) : "))
theta2 = float(input("Choisir l'angle du vecteur-vitesse du deuxième véhicule p/r à l'horizontal' (en degré) : "))

#définissons les vecteurs vitesse

v1 = np.array([u1*m.cos(theta1*m.pi/180),u1*m.sin(theta1*m.pi/180)])
v2 = np.array([u2*m.cos(theta2*m.pi/180),u2*m.sin(theta2*m.pi/180)])
m3 = m1 + m2

v3 = 1/m3*(m1*v1 + m2*v2)

print(v1)
print(v2)
print(v3)

v3x = v3[0]
v3y = v3[1]
v3 = m.sqrt(v3x**2 + v3y**2)
theta3 = np.arctan(v3y/v3x)*180/m.pi

print('la composante x de v3 est ', v3x)
print('la composante y de v3 est ', v3y)
print('les deux voiture part avec une vitesse ', v3, ' avec un angle ',theta3, ' degré par rapport à horizontal')
\end{lstlisting}




\end{document}